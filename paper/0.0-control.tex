% Options for packages loaded elsewhere
\PassOptionsToPackage{unicode}{hyperref}
\PassOptionsToPackage{hyphens}{url}
%
\documentclass[
]{article}
\usepackage{lmodern}
\usepackage{amsmath}
\usepackage{ifxetex,ifluatex}
\ifnum 0\ifxetex 1\fi\ifluatex 1\fi=0 % if pdftex
  \usepackage[T1]{fontenc}
  \usepackage[utf8]{inputenc}
  \usepackage{textcomp} % provide euro and other symbols
  \usepackage{amssymb}
\else % if luatex or xetex
  \usepackage{unicode-math}
  \defaultfontfeatures{Scale=MatchLowercase}
  \defaultfontfeatures[\rmfamily]{Ligatures=TeX,Scale=1}
\fi
% Use upquote if available, for straight quotes in verbatim environments
\IfFileExists{upquote.sty}{\usepackage{upquote}}{}
\IfFileExists{microtype.sty}{% use microtype if available
  \usepackage[]{microtype}
  \UseMicrotypeSet[protrusion]{basicmath} % disable protrusion for tt fonts
}{}
\makeatletter
\@ifundefined{KOMAClassName}{% if non-KOMA class
  \IfFileExists{parskip.sty}{%
    \usepackage{parskip}
  }{% else
    \setlength{\parindent}{0pt}
    \setlength{\parskip}{6pt plus 2pt minus 1pt}}
}{% if KOMA class
  \KOMAoptions{parskip=half}}
\makeatother
\usepackage{xcolor}
\IfFileExists{xurl.sty}{\usepackage{xurl}}{} % add URL line breaks if available
\IfFileExists{bookmark.sty}{\usepackage{bookmark}}{\usepackage{hyperref}}
\hypersetup{
  pdftitle={The Value of High-Excess Commercial Insurance (working title)},
  pdfauthor={Zhenkai Zhu; Rajesh Sahasrabuddhe},
  hidelinks,
  pdfcreator={LaTeX via pandoc}}
\urlstyle{same} % disable monospaced font for URLs
\usepackage[margin=1in]{geometry}
\usepackage{longtable,booktabs}
\usepackage{calc} % for calculating minipage widths
% Correct order of tables after \paragraph or \subparagraph
\usepackage{etoolbox}
\makeatletter
\patchcmd\longtable{\par}{\if@noskipsec\mbox{}\fi\par}{}{}
\makeatother
% Allow footnotes in longtable head/foot
\IfFileExists{footnotehyper.sty}{\usepackage{footnotehyper}}{\usepackage{footnote}}
\makesavenoteenv{longtable}
\usepackage{graphicx}
\makeatletter
\def\maxwidth{\ifdim\Gin@nat@width>\linewidth\linewidth\else\Gin@nat@width\fi}
\def\maxheight{\ifdim\Gin@nat@height>\textheight\textheight\else\Gin@nat@height\fi}
\makeatother
% Scale images if necessary, so that they will not overflow the page
% margins by default, and it is still possible to overwrite the defaults
% using explicit options in \includegraphics[width, height, ...]{}
\setkeys{Gin}{width=\maxwidth,height=\maxheight,keepaspectratio}
% Set default figure placement to htbp
\makeatletter
\def\fps@figure{htbp}
\makeatother
\setlength{\emergencystretch}{3em} % prevent overfull lines
\providecommand{\tightlist}{%
  \setlength{\itemsep}{0pt}\setlength{\parskip}{0pt}}
\setcounter{secnumdepth}{5}
\ifluatex
  \usepackage{selnolig}  % disable illegal ligatures
\fi

\title{The Value of High-Excess Commercial Insurance (working title)}
\author{Zhenkai Zhu \and Rajesh Sahasrabuddhe}
\date{December 28, 2020}

\begin{document}
\maketitle
\begin{abstract}
High excess layer property-casualty insurance serves as a source of
contingent capital for the insured. That is, instead of setting aside
capital to fund rare but extreme events, firms will purchase insurance.
High layer excess insurance, in particular, serves this purpose. Firms
rarely access these layers, and the expected value of claims in those
layers is minimal. As such, the insurance policy's primary value is that
it allows the firm to deploy the capital that it would otherwise need to
set aside to weather an extreme loss event. This paper describes an
approach to measure this value.
\end{abstract}

\hypertarget{introduction}{%
\section{Introduction}\label{introduction}}

Commercial insurance programs are often depicted as ``towers.'' (See
Figure \ref{fig:tower}) The lowest floors of the tower are typically
comprised of risk retained by the insured. The middle floors include
commercially insured working layers. High-layer excess insurance occupy
the penthouse.

Each of these layers has a cost to it.

\begin{itemize}
\tightlist
\item
  The cost of the lowest layers will have a reasonable degree of
  predictability. The cost for these layers will be principally be the
  expected loss with a minimal capital charge for volatility. There may
  also be frictional costs associated with risk retention such as
  increased audit and actuarial fees and claims costs from legal counsel
  and third party claim administrators.
\item
  The middle layers provide coverage for claims related to events that
  occur with a degree of regularity. The amount of claims may not be
  reasonably predictable in a single year but it would be predictable
  over a longer-term time horizon. They insurance coverage provides
  value to the insured by smoothing claim volatility over time. There
  may be multiple insurers in the higher working layers.
\item
  Insureds generally do not expect access coverage from the highest
  layers. The expected claims will likely be close to \$0. For the
  insured-insurer relationship, this layer provides some smoothing of
  claims volatility over time. However, insurance in this layer smooth
  claim volatility across risks. As a result, there are often multiple
  insurers in these layers, each with a pro-rata share.
\end{itemize}

\begin{figure}
\centering
\includegraphics{0.0-control_files/figure-latex/insurance tower-1.pdf}
\caption{\label{fig:tower}Example Insurance Tower}
\end{figure}

\hypertarget{measuring-value}{%
\section{Measuring Value}\label{measuring-value}}

Despite insureds not expected to access higher layers and with expected
claim amounts being minimal, premiums in these layers can be
significant. In this paper we explore the \emph{value} of the coverage
in this layer. Measuring this value provides the insured a basis upon
which to assess the premium established by the insurer.

To assess the value, we should understand that the primary benefit of
this layer is not in the expected loss but the protection of capital
that this layer of insurance provides to the insured. Indeed, in this
layer insureds often consider the insurance as part of its capital
structure rather than as an operational cost.

\newpage

Below, we present notation that we will use in this paper.

\begin{longtable}[]{@{}cl@{}}
\toprule
\endhead
\begin{minipage}[t]{(\columnwidth - 1\tabcolsep) * \real{0.05}}\centering
\(C\)\strut
\end{minipage} &
\begin{minipage}[t]{(\columnwidth - 1\tabcolsep) * \real{0.95}}\raggedright
Total \textbf{capital} for the insured\strut
\end{minipage}\tabularnewline
\begin{minipage}[t]{(\columnwidth - 1\tabcolsep) * \real{0.05}}\centering
\(C_w\)\strut
\end{minipage} &
\begin{minipage}[t]{(\columnwidth - 1\tabcolsep) * \real{0.95}}\raggedright
\textbf{Working capital} The capital that the available for the insured
to conduct operations\strut
\end{minipage}\tabularnewline
\begin{minipage}[t]{(\columnwidth - 1\tabcolsep) * \real{0.05}}\centering
\(C_m\)\strut
\end{minipage} &
\begin{minipage}[t]{(\columnwidth - 1\tabcolsep) * \real{0.95}}\raggedright
\textbf{Minimum capital} The minimum capital required for the insured to
operate.\strut
\end{minipage}\tabularnewline
\begin{minipage}[t]{(\columnwidth - 1\tabcolsep) * \real{0.05}}\centering
\(A\)\strut
\end{minipage} &
\begin{minipage}[t]{(\columnwidth - 1\tabcolsep) * \real{0.95}}\raggedright
The \textbf{attachment} point of the high excess layer insurance
policy\strut
\end{minipage}\tabularnewline
\begin{minipage}[t]{(\columnwidth - 1\tabcolsep) * \real{0.05}}\centering
\(W\)\strut
\end{minipage} &
\begin{minipage}[t]{(\columnwidth - 1\tabcolsep) * \real{0.95}}\raggedright
The \textbf{width} of the high excess layer insurance policy\strut
\end{minipage}\tabularnewline
\begin{minipage}[t]{(\columnwidth - 1\tabcolsep) * \real{0.05}}\centering
\(W\) xs \(A\)\strut
\end{minipage} &
\begin{minipage}[t]{(\columnwidth - 1\tabcolsep) * \real{0.95}}\raggedright
Common terminology to refer to the amount of coverage provided by the
excess policy\strut
\end{minipage}\tabularnewline
\begin{minipage}[t]{(\columnwidth - 1\tabcolsep) * \real{0.05}}\centering
\(P\)\strut
\end{minipage} &
\begin{minipage}[t]{(\columnwidth - 1\tabcolsep) * \real{0.95}}\raggedright
\textbf{Premium} The amount charged by an insurer to accept risk\strut
\end{minipage}\tabularnewline
\begin{minipage}[t]{(\columnwidth - 1\tabcolsep) * \real{0.05}}\centering
\(N\), \(n\)\strut
\end{minipage} &
\begin{minipage}[t]{(\columnwidth - 1\tabcolsep) * \real{0.95}}\raggedright
Random variable for the \textbf{number of claims} in excess of \(A\) and
realizations of that random variable.\strut
\end{minipage}\tabularnewline
\begin{minipage}[t]{(\columnwidth - 1\tabcolsep) * \real{0.05}}\centering
\(X\), \(x\)\strut
\end{minipage} &
\begin{minipage}[t]{(\columnwidth - 1\tabcolsep) * \real{0.95}}\raggedright
Random variable for claim values in excess of \(A\) (i.e., \$X\$
conditioned on \(X > A\)), and realizations of that random
variable.\strut
\end{minipage}\tabularnewline
\begin{minipage}[t]{(\columnwidth - 1\tabcolsep) * \real{0.05}}\centering
\(S\)\strut
\end{minipage} &
\begin{minipage}[t]{(\columnwidth - 1\tabcolsep) * \real{0.95}}\raggedright
In its enterprise risk management framework, the claim amount in the
\(W\) x \(A\) layer for which the insured will satisfy its obligations
and continue to function. That is, for \(X\) \textgreater{} \(S\), the
insured might need to declare bankruptcy rather than continue to
function.

Readers should understand that S may be greater than \(C - C_m\) as the
firm may be able to recapitalize to \(C_m\) after experiencing a claim
of size S which results in capital temporarily dropping below
\(C_m\).\strut
\end{minipage}\tabularnewline
\begin{minipage}[t]{(\columnwidth - 1\tabcolsep) * \real{0.05}}\centering
\(R\)\strut
\end{minipage} &
\begin{minipage}[t]{(\columnwidth - 1\tabcolsep) * \real{0.95}}\raggedright
\textbf{Revenue} for the product or service sold by the insured\strut
\end{minipage}\tabularnewline
\begin{minipage}[t]{(\columnwidth - 1\tabcolsep) * \real{0.05}}\centering
\(Z\)\footnote{We use \(Z\) for expenses so as to not create confusion
  with the expectations operator, \(E\)}\strut
\end{minipage} &
\begin{minipage}[t]{(\columnwidth - 1\tabcolsep) * \real{0.95}}\raggedright
All expenses associated with the production of the product of service
sold by the insured other than insurance premiums.\strut
\end{minipage}\tabularnewline
\begin{minipage}[t]{(\columnwidth - 1\tabcolsep) * \real{0.05}}\centering
\(R_x\)\strut
\end{minipage} &
\begin{minipage}[t]{(\columnwidth - 1\tabcolsep) * \real{0.95}}\raggedright
Within the insured's pricing model for its product or services, the
portion of revenue that represents recovery of the cost of loss events
from its customer.\strut
\end{minipage}\tabularnewline
\bottomrule
\end{longtable}

\hypertarget{simplifying-assumptions}{%
\subsection{Simplifying Assumptions}\label{simplifying-assumptions}}

In this paper, we present the measurement of the value of insurance all
else being equal.We use the following simplifying assumptions to focus
on value:

\begin{itemize}
\item
  We recognize that, in practice, revenue (\(R\)) and expenses (\(Z\))
  are subject to variation. However, in this paper, we will treat these
  as fixed and known quantities.
\item
  We also assume market prices for the firm's products, and revenue as a
  result, are not affected by whether the firm elects to purchase
  insurance.
\item
  We assume that all firms that sell the same product have the sames
  cost structure.
\item
  We assume that investors demand a return on capital commensurate with
  risk. That is, investors will require a lower return on capital from
  the insurance buyer.
\end{itemize}

\hypertarget{a-pedagogical-example}{%
\subsection{A Pedagogical Example}\label{a-pedagogical-example}}

To support the reader's understanding of the concepts presented in this
paper, we offer the following pedagogical example. We offer the
hypothetical airline ``PC Air.''

\begin{itemize}
\item
  PC Air leases a single plane.
\item
  PC Air's offers cargo transport between New York and Los Angeles. That
  route requires PC Air's plane to fly over several major ground
  structures (such as warehouses). PC Air only flies overnight when
  these structures are not occupied.
\item
  PC Air developed a technology that allows it to fly its plan remotely
  (no pilot).
\item
  Airplane crashes are exceeding rare.
\item
  PC Air's enterprise risk management targets solvency if its plan were
  to crash in a remote area. That is, it would be able to reimburse its
  cargo customers and pay for the owner of the cost of the plane. Its
  remaining capital would be sufficient for it to lease a new plane. If
  PC Air's plane were to cause damage to ground structures, it would
  declare bankruptcy.
\end{itemize}

Throughout this paper, we will refer back to the example of PC Air to
help the reader understand that how these concepts work in practice.

\hypertarget{the-no-insurance-base-case}{%
\subsection{The No Insurance Base
Case}\label{the-no-insurance-base-case}}

We start by considering the no insurance base case. For simplicity, we
assume that only one claim can occur during the policy period.

Under these conditions:

\begin{itemize}
\item
  \(C_w = C_0 - S\) The firm's (beginning) working capital is equal to
  its total capital less the amount that it elects to set aside for
  potential claims event.
\item
  The change in capital (\(\Delta C\)) depends upon whether a loss event
  occurs.

  \begin{itemize}
  \item
    No loss event (\(n\) = 0): \(\Delta C = R - Z\)
  \item
    Loss event occurs (\(n\) = 1): \(\Delta C = R - Z - X\). Critically,
    we recognize that a portion of the distribution of \(\Delta C\) is
    greater than \(C - C_m\). For these realziations of \$X\$, the firm
    must declare bankruptcy.
  \end{itemize}
\end{itemize}

\hypertarget{the-insurance-case}{%
\subsection{The Insurance Case}\label{the-insurance-case}}

If the firm elects to purchase insurance, then:

\begin{itemize}
\item
  \(C_w = C_I - P\) The insured will need to pay premium but it no
  longer needs to set aside capital to cover potential claim events.
\item
  The change in capital is no longer depends on whether a loss occurs:
  \(\Delta C = R - Z - P\).
\end{itemize}

\hypertarget{capital-relationships}{%
\section{Capital Relationships}\label{capital-relationships}}

We now consider the firm's capital. We present the development of the
capital structure below:

\begin{eqnarray[*]}
\text{Capital} &= \text{Assets} - \text{Liabilities}
\text{Assets} &= \text{Fixed Assets} - \text{Current Assets}

\end{eqnarray[*]}

\text{Assets} = Capital + Liabilities (Is this correct? I changed
``Equity'' to ``Capital'')

Assets = Fixed Assets + Current Assets

Capital = Equity Capital + Debt Capital

Capital = Fixed Assets + Net Working Capital (Current Assets Current
Liabilities)

\includegraphics{0.0-control_files/figure-latex/unnamed-chunk-7-1.pdf}

\hypertarget{can-the-insurer-buyer-maintain-a-lower-level-of-capital}{%
\subsubsection{Can the insurer buyer maintain a lower level of
capital?}\label{can-the-insurer-buyer-maintain-a-lower-level-of-capital}}

Under no insurance case, Net Working Capital includes \(C_w\) and
Reserve set aside for potential claims event.

About S, in the notation description, it seems to represent the largest
affordable loss amount, however, in the first formula of 2.1, it
represents the amount set aside for potential claims event (which I
thought is the reserve)

So I want to use notation \(Res\) to represent the Reserve to
distinguish them.

I am trying to explain my concept in the following plot.

Back to the question, Insurance can protect losses, revenue is treated
as fixed and known quantity, so the companies with insurance would not
suffer from any unpredictable loss except the coverage has a upper
limit.Therefore, the capital would not decrease to a low level under our
assumption.

Insurance can help company release more \(C_w\), but can not change
other capital requirement.

\hypertarget{how-does-that-volatility-of-the-three-situations-above-affect-the-return-on-capital-demanded-by-investors}{%
\subsubsection{3. How does that volatility of the three situations above
affect the return on capital demanded by
investors?}\label{how-does-that-volatility-of-the-three-situations-above-affect-the-return-on-capital-demanded-by-investors}}

To measure the volatility, we can use \(var(\Delta C)\) or use the delta
(\(\frac{d\Delta C}{dX}\)). Obviously, under the insurance case, if the
only variable is X and coverage has no upper limit, both these 2
measurements would be 0.

To connect the volatility and required return of capital, we can apply
the efficient frontier. The efficient frontier is the set of optimal
portfolios that offer the highest expected return for a defined level of
risk or the lowest risk for a given level of expected return.

\end{document}
