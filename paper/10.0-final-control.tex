% Options for packages loaded elsewhere
\PassOptionsToPackage{unicode}{hyperref}
\PassOptionsToPackage{hyphens}{url}
%
\documentclass[
]{article}
\usepackage{lmodern}
\usepackage{amsmath}
\usepackage{ifxetex,ifluatex}
\ifnum 0\ifxetex 1\fi\ifluatex 1\fi=0 % if pdftex
  \usepackage[T1]{fontenc}
  \usepackage[utf8]{inputenc}
  \usepackage{textcomp} % provide euro and other symbols
  \usepackage{amssymb}
\else % if luatex or xetex
  \usepackage{unicode-math}
  \defaultfontfeatures{Scale=MatchLowercase}
  \defaultfontfeatures[\rmfamily]{Ligatures=TeX,Scale=1}
\fi
% Use upquote if available, for straight quotes in verbatim environments
\IfFileExists{upquote.sty}{\usepackage{upquote}}{}
\IfFileExists{microtype.sty}{% use microtype if available
  \usepackage[]{microtype}
  \UseMicrotypeSet[protrusion]{basicmath} % disable protrusion for tt fonts
}{}
\makeatletter
\@ifundefined{KOMAClassName}{% if non-KOMA class
  \IfFileExists{parskip.sty}{%
    \usepackage{parskip}
  }{% else
    \setlength{\parindent}{0pt}
    \setlength{\parskip}{6pt plus 2pt minus 1pt}}
}{% if KOMA class
  \KOMAoptions{parskip=half}}
\makeatother
\usepackage{xcolor}
\IfFileExists{xurl.sty}{\usepackage{xurl}}{} % add URL line breaks if available
\IfFileExists{bookmark.sty}{\usepackage{bookmark}}{\usepackage{hyperref}}
\hypersetup{
  pdftitle={On the Value of High-Excess Commercial Insurance},
  pdfauthor={Rajesh Sahasrabuddhe; Zhenkai Zhu},
  hidelinks,
  pdfcreator={LaTeX via pandoc}}
\urlstyle{same} % disable monospaced font for URLs
\usepackage[margin=1in]{geometry}
\usepackage{graphicx}
\makeatletter
\def\maxwidth{\ifdim\Gin@nat@width>\linewidth\linewidth\else\Gin@nat@width\fi}
\def\maxheight{\ifdim\Gin@nat@height>\textheight\textheight\else\Gin@nat@height\fi}
\makeatother
% Scale images if necessary, so that they will not overflow the page
% margins by default, and it is still possible to overwrite the defaults
% using explicit options in \includegraphics[width, height, ...]{}
\setkeys{Gin}{width=\maxwidth,height=\maxheight,keepaspectratio}
% Set default figure placement to htbp
\makeatletter
\def\fps@figure{htbp}
\makeatother
\setlength{\emergencystretch}{3em} % prevent overfull lines
\providecommand{\tightlist}{%
  \setlength{\itemsep}{0pt}\setlength{\parskip}{0pt}}
\setcounter{secnumdepth}{5}
\ifluatex
  \usepackage{selnolig}  % disable illegal ligatures
\fi

\title{On the Value of High-Excess Commercial Insurance}
\author{Rajesh Sahasrabuddhe \and Zhenkai Zhu}
\date{February 21, 2021}

\begin{document}
\maketitle
\begin{abstract}
Insurance serves both as a source of loss reimbursement and contingent
capital. The expected value of claims (i.e., loss reimbursement value)
for the insured in high excess layers is minimal, so the principal value
of insurance in those layers is as a source of contingent capital. This
paper describes an approach to measure this source of value.
\end{abstract}

\raggedright

\hypertarget{introduction}{%
\section{Introduction}\label{introduction}}

In this paper, we provide a basis to measure the value provided by
high-excess layer insurance policies. Our focus in this paper is the
highest excess layers purchased by large commercial firms. Insured can
compare the value and the premium for the insurance policy to understand
whether the insurance purchase is accretive.

\hypertarget{commercial-insurance-programs}{%
\subsection{Commercial Insurance
Programs}\label{commercial-insurance-programs}}

Commercial insurance programs are often depicted as ``towers'' (See
Figure \ref{fig:tower}). Extending this analogy, risk retained by the
insured typically comprised the lowest floors of the tower . The middle
floors include commercially-insured working layers. High excess layer
insurance resides in the penthouse.

Each of these layers has a cost to it. Insurers establish premiums which
are sufficient to cover claim costs, expenses and return a profit to its
shareholder.

The profit provision will be proportional to the volatility of the risk.
That is, the required supporting insurer capital is directly
proportional to the volatility of the business. As a result, more
volatile business will require additional profit to satisfy the
insurer's shareholder. Consistent with this notion, we ignore expenses
and refer to the profit provision as a capital charge in describing the
cost for various insurance layers below.

\begin{itemize}
\item
  The cost of the lowest layers will have a reasonable degree of
  predictability. Because costs are predictable, the value of insurance
  is minimal, and insureds will, therefore, typically retain this
  risk.\footnote{The insured may realize value though specialized risk
    or claims management services provided by the insurer.} The cost for
  these layers will principally be the expected loss with a minimal
  capital charge for volatility. There may also be frictional costs
  associated with risk retention, such as increased audit and actuarial
  fees and claims costs from legal counsel and third-party claim
  administrators.
\item
  The middle layers provide coverage for claims related to events that
  occur with a degree of regularity. The amount of claims may not be
  reasonably predictable in a single year but would be predictable over
  a longer time horizon. The insurance coverage provides value to the
  insured by smoothing claim volatility \emph{over time}. There may be
  multiple insurers in the higher working layers.
\item
  Insureds generally do not expect to access coverage from the highest
  layers. The expected claims will likely be close to \$0. The variance
  of claims in this layer, however, may be significant. Insurance in
  this layer smooths claims volatility \emph{over risks and time}. As a
  result of the need to smooth volatility across risks, there are often
  multiple insurers in these layers, each with a pro-rata share. These
  high excess layers are the focus of this paper.
\end{itemize}

\begin{figure}
\centering
\includegraphics{10.0-final-control_files/figure-latex/insurance tower-1.pdf}
\caption{\label{fig:tower}Example Insurance Tower}
\end{figure}

\hypertarget{the-value-of-high-excess-layers}{%
\subsection{The Value of High Excess
Layers}\label{the-value-of-high-excess-layers}}

Despite minimal expected claim values, premiums in these layers can be
significant. Sometimes buyers refer to these layers as ``sleep
insurance''; that is, the insurance allows the buyer to sleep at night.
In this paper, we describe an approach that the buyer can consider to
understand whether there is \emph{financial value}\footnote{That is, the
  value beyond the restful night of sleep.} to the insurance
transaction.

To assess the value, we should understand that the primary benefit of
this layer is not in the loss reimbursement but the protection of
capital that this layer of insurance provides to the insured.

\begin{itemize}
\item
  Insureds often consider the insurance in this layer as part of its
  capital structure rather than as an operational cost. The insurance
  allows the firm to deploy the capital that it would otherwise need to
  set aside to weather an extreme loss event.
\item
  In addition to this protection of capital, insurance will also reduce
  earning volatility and potentially improve the attractiveness of the
  firm to investors.
\end{itemize}

In this paper, we will quantify these benefits of insurance.

\hypertarget{literature-search}{%
\subsection{Literature Search}\label{literature-search}}

Most research related to capital in the insurance transaction focuses on
the \emph{insurer's} return on capital rather than the protection of the
\emph{insured's} capital. The limited research related to the insured's
preference focuses on utility theory (as a function of wealth), and that
research is generally presented in the context of personal insurance
coverages (such as homeowners).

Capital protection and wealth maximization are related concepts, and
utility functions (generically) provide a basis to assess alternatives.
This is not dissimilar to the approach that we present in this paper.
However, there are differences to the research we present:

\begin{itemize}
\item
  Utility functions are typically abstract, whereas the measurement
  approach we present in this paper is more specifically defined.
\item
  We frame our approach in the context of commercial insurance and
  present two value considerations for a commercial insured.
\end{itemize}

We list the papers that we reviewed in Appendix \ref{literature-review}.

\hypertarget{presentation-outline}{%
\subsection{Presentation Outline}\label{presentation-outline}}

In this paper, we consider the two primary sources of value for the
insured:

\begin{itemize}
\item
  The ability to deploy capital that it would have otherwise had to set
  aside for an insurable event. We present our review of this source of
  value in Section \ref{capital-deployment}.
\item
  Reduced volatility of earnings and the resulting reduction in required
  return. We present our review of this source of value in Section
  \ref{risk-adjustment-to-required-return}.
\end{itemize}

We use the Sharpe Ratio in our analysis in Section
\ref{risk-adjustment-to-required-return}. As we will discover from this
analysis, the measured value from the Sharpe Ratio will \emph{include}
the value related to capital deployment.

\hypertarget{capital-deployment}{%
\section{Capital Deployment}\label{capital-deployment}}

In this section, we describe how high-excess insurance allows the
insured to deploy capital that it would otherwise need to set aside to
weather an extreme loss event.

The analysis we present does not relate to the entire capital structure
of the firm but rather only that portion allocated to activities that
are subject to the risk of claims. In order to measure the value of
insurance related to the firm's ability to deploy that capital, we
compare rates of return when the firm ``funds'' the claims risk without
insurance (i.e., with capital) and with insurance.

\hypertarget{funding-risk-using-capital}{%
\subsection{Funding Risk using
Capital}\label{funding-risk-using-capital}}

We first present an analysis to demonstrate \emph{conceptually} why a
firm would find it necessary to set aside capital; we do not intend this
analysis to be a prescriptive approach for determining the amount of
capital to set aside. In our review, we assume the following:

\begin{itemize}
\item
  There is a fixed amount of capital, \(C\), available to the firm.
\item
  The firm requires a minimum level of capital, \(C_{min}\), to continue
  as a going concern.
\item
  We denote deployed capital, i.e., the working capital, as \(C_w\).
\item
  The firm is subject to the loss events that result in aggregate claim
  amounts, \(X\). We present the distribution of claim amounts in Figure
  \ref{fig:clm-dist}\footnote{This is an illustrative claim distribution
    that we present to support further development of value measurement.}.

  \begin{itemize}
  \item
    For simplicity, we assume information symmetry as respects \(X\).
    That is, the insurer and the insured use the same distribution for
    \(X\).
  \item
    We denote the maximum probable claim amount as \(X_{max}\).
  \item
    The scale of \(X\) is a function of \(C_w\).
  \end{itemize}
\end{itemize}

\begin{figure}
\centering
\includegraphics{10.0-final-control_files/figure-latex/claim-distribution-1.pdf}
\caption{\label{fig:clm-dist}Illustrative Claim (X) Distribution}
\end{figure}

\begin{itemize}
\item
  The firm has an enterprise risk management (ERM) strategy underlying
  its capital allocation. Its ERM strategy dictates that it can absorb a
  claim at the \(p^{th}\) percentile of the distribution of \(X\), which
  we denote as \(X_p\).
\item
  The firm is able to generate a return on capital of \(r_c\). For
  purposes of determining the value of insurance, we use a simplifying
  assumption that \(r_c\) is fixed. That is, \(E(r_c)=r_c\) and
  \(Var(r_c)=0\).
\item
  We denote the risk-free rate \(r_f\).
\end{itemize}

\hypertarget{rate-of-return-without-insurance}{%
\subsection{Rate of Return without
Insurance}\label{rate-of-return-without-insurance}}

When a firm retains risk, it must allocate capital for that risk. The
firm may allocate that capital with or without a formal analysis as to
the amount of capital that it must set aside. In this section, we
present an example of actions that a firm with an ERM strategy may take
when faced with risk. We recognize that not all firms will allocate
capital using this type of analysis.

The firm's ERM strategy will require that it reserve capital for the
possibility of that it experiences a capital depletion event. We denote
this reserve as \(C_{x}\) to indicate that the reserve is a segregation
of capital to absorb realizations of \(X\). We present that capital
allocation approach in Figure \ref{fig:cap-deploy}.

\begin{figure}
\centering
\includegraphics{10.0-final-control_files/figure-latex/No Insurnace-1.pdf}
\caption{\label{fig:cap-deploy}Capital Deployment}
\end{figure}

Using the naught superscript to represent this base case, the rate of
return is then as follows:

\begin{equation}
R^0 = \dfrac{r_c \times C_{w}^{0} + r_f \times C_{x}^{0} - X}{C} \label{eqn:return-base}
\end{equation}

The numerator is the sum of the return on working capital and the return
on the reserve, less the value of the claim amount. The inclusion of the
random variable \(X\) indicates that the firm's return is a function of
the realized value of the loss event.

We note, however, that the expected value of \(X\) in high excess layers
is nearly 0. That is, \(E[X] \approx 0\). As s result, taking
expectations, we have:

\begin{equation}
E[R^0] = \dfrac{r_c \times C_{w}^{0} + r_f \times C_{x}^{0}}{C} \label{eqn:exp-return-base}
\end{equation}

\hypertarget{rate-of-return-with-insurance}{%
\subsection{Rate of Return with
Insurance}\label{rate-of-return-with-insurance}}

The firm may alternatively elect to purchase insurance to cover a
portion of the cost of the loss event. We develop that rate of return in
this section. We use the following notation in our rate of return
equation:

\begin{itemize}
\item
  \(X_{ret}\) represents the portion of the distribution of claims
  values that the firm retains. We allow \(X_{ret}\) to vary between 0
  (equal to the no insurance case) and \(X_m\) (in which case, the firm
  fully transfers the risk). We observe that if retention will not
  reduce capital below \(C_m\), i.e., \(0 < X_{ret} < (C_w - C_m)\),
  then no capital need to be set aside to cover loss events, i.e.,
  \(C_x = 0\).
\item
  We use superscript \(I\) to represent the ``insurance case.'' and
  \(P^{I}\) for premium (in the insurance case). This is consistent with
  the \(C_{x}^{0}\) notation that we use for capital. (They both
  represent approaches to fund risk.)
\end{itemize}

The rate of return in the insurance case is as follows:

\begin{equation}
R^I = \dfrac{r_c \times (C - C_{x}^{I} - P^{I}) + r_f \times C_{x}^{I} - X_{ret} - P^{I}}{C} \label{eqn:return-ins}
\end{equation}

The rational firm (where \(r_c > r_f\)\footnote{Without this condition
  there is no economic value to the firm's existence.}) purchases
insurance such that it need not set aside any capital. Therefore,
\(C_{x}^{I} = 0\) and \(C_{w}^{I} = C - P^{I}\).

As we noted above, \(E[X] \approx 0\). Therefore,
\(E[X_{ret}] \approx 0\) and \(E[X - X_{ret}] \approx 0\). We can then
simplify our rate of return equation to:

\begin{equation}
E[R^I] = \dfrac{r_c \times (C - P^{I}) - P^{I}}{C} \label{eqn:exp-return-ins}
\end{equation}

We recognize that \(P^{I}\) is the sum of the expected loss
\(E[X - X_{ret}]\) and an insurance charge. As noted, the expected loss
is approximately \(0\). The insurance charge compensates the insurer for
underwriting the exposure and provides for a return on its capital.
Therefore we expect the insurance charge to be a \emph{function} of
\(Var(X - X_{ret})\).\footnote{We recognize that the insurer has the
  ability to diversify this volatility across risks. As a result the
  function referenced can result in a premiums that creates value.} In
our construct, the prospective insured only observes premiums quoted
under various options. As such, it is not concerned with the function
the insurer uses to develop its risk charge, but it is aware that the
premium is almost entirely comprised of the risk charge.

\hypertarget{capital-deployment-value-creation-equation}{%
\subsection{Capital Deployment Value Creation
Equation}\label{capital-deployment-value-creation-equation}}

The insurance transaction creates value when the expected return for the
insurance buyer exceeds the expected return without insurance.
Specifically, we compare the expected returns from equation
(\ref{eqn:exp-return-base}) and equation (\ref{eqn:exp-return-ins}). We
observe that the purchase of insurance creates value when:

\begin{align}
\nonumber E[R^I] &> E[R^0] \\
\nonumber r_c \times (C - P^{I}) - P^{I} &> r_c \times C_{w}^{I} + r_f \times C_{x}^{0} \\
\cdots &> \cdots\\
P^{I} &< C_{x}^{0} \times \dfrac{r_c - r_f}{1 + r_c} \label{eqn:cap-deploy}
\end{align}

We include the complete algebraic derivation of equation
\ref{eqn:cap-deploy} in Appendix
\ref{capital-deployment-value-creation-equation}. Equation
(\ref{eqn:cap-deploy}) has a straightforward, intuitive interpretation
that the insurance transaction creates value when the premium is less
than the excess return on capital that would be earned on the reserve
reduced for the return if the premium amount were also deployed as
capital.

\hypertarget{risk-adjustment-to-required-return}{%
\section{Risk Adjustment to Required
Return}\label{risk-adjustment-to-required-return}}

The second source of value created by insurance results from the
reduction in the volatility of returns for insurance buyers. The value
creation results from the lower required return for firms with reduced
earning volatility.

The Sharpe Ratio relates the risk premium required (numerator) to return
volatility (denominator). More specifically, the Sharpe Ratio indicates
the risk premium required for every unit of volatility. As we did in
measuring the value through capital deployment, we first calculate the
Sharpe Ratio without insurance (i.e., the base case) and then compare to
the Sharpe Ratio for the insurance buyer.

\[
\text{Sharpe Ratio} = \frac{E(R)-r_f}{\sigma_{R}}
\]

As we are concerned with the marginal value created by the insurance
transaction, we can calculate \(E(R)\), \(Var(R)\) in this section under
the following assumptions:

\begin{itemize}
\tightlist
\item
  The risk-free rate, \(r_f\), is fixed and\\
\item
  \(X\) and \(r_c\) are independent.
\end{itemize}

\hypertarget{sharpe-ratio-without-insurance}{%
\subsection{Sharpe Ratio without
insurance}\label{sharpe-ratio-without-insurance}}

As with the discussion in Section \ref{capital-deployment}, we use the
naught superscript to represent the base case.

Recall that we developed the expected rate of return in Section
\ref{rate-of-return-without-insurance}. We rewrite equation
\ref{eqn:exp-return-base} as presented below.

\[
\begin{aligned}
E[R^{0}] &= E\left[\dfrac{r_c \times C_w^{0} + r_f \times C_{x}^{0}}{C}\right]\\
&=\dfrac{(C-C_{x}^{0})\times r_c + r_f \times C_{x}^{0}}{C}\\
\end{aligned}
\]

We start with equation (\ref{eqn:return-base}) and develop the variance
of returns under the simplifying conditions described at the beginning
of this section.

\[
\begin{aligned}
Var[R^{0}] &= Var\left[\dfrac{r_c \times C_w^{0} + r_f \times C_{x}^{0} - X}{C}\right]\\
&=\dfrac{Var(X)}{C^2}
\end{aligned}
\] We can now calculate the Sharpe Ratio without insurance.

\begin{align}
\nonumber \text{Sharpe Ratio}^0 &=\frac{E(R_0)-r_f}{\sigma_{R_0}}\\
\nonumber &=\dfrac{\left[\dfrac{(C-C_{x}^{0})\times r_c + r_f \times C_{x}^{0}-r_f \times C}{C}\right]}{\left[\dfrac{Var(X)}{C^2}\right]^{0.5}}\\
           &=\frac{(C-C_{x}^{0})\times (r_c-r_f)}{SD(X)} \label{eqn:sharpe-base}
\end{align}

\hypertarget{sharpe-ratio-with-insurance}{%
\subsection{Sharpe Ratio with
Insurance}\label{sharpe-ratio-with-insurance}}

We can also calculate the Sharpe Ratio for the insurance buyer, using
the superscript \(I\) to represent this case. As with the no insurance
case, we start with equation (\ref{eqn:exp-return-ins}) from
Section\ref{rate-of-return-without-insurance}: \[
\begin{aligned}
E[R^{I}] &= E[\dfrac{r_c \times (C - P^{I}) - P^{I}}{C}]
\end{aligned}
\]

We start with equation (\ref{eqn:return-ins}) and develop the variance
under the simplifying conditions described at the beginning of this
section. \[
\begin{aligned}
Var[R^{I}] &= Var(\dfrac{r_c \times (C - P^{I}) + r_f \times C_{x}^{I} - X_{ret} - P^{I}}{C})\\
Var[R^{I}] &=\dfrac{Var(X_{ret})}{C^2}
\end{aligned}
\]

Now we calculate the Sharpe Ratio under the insurance case:

\begin{align}
\nonumber \text{Sharpe Ratio}^I &= \dfrac{E(R_I)-r_f}{\sigma_{R^0}}\\
\nonumber &=\dfrac{\left[\dfrac{(C - P^{I}) \times r_c-P^{I} - C \times r_f}{C}\right]}{\left[\dfrac{Var(X_{ret})}{C^2}\right]^{0.5}}\\
&=\dfrac{C\times (r_c-r_f)-P^{I} \times(1 + r_c)}{SD(X_{ret})} \label{eqn:sharpe-ins}
\end{align}

\hypertarget{complete-value-creation-equation}{%
\subsection{Complete Value Creation
Equation}\label{complete-value-creation-equation}}

The insurance purchase created value when it results in an increase in
the Sharpe Ratio. We can use equation (\ref{eqn:sharpe-base}) and
equation (\ref{eqn:sharpe-ins}) to calculate the maximum premium that
results in value creation. We present that equation below. We include
the complete algebraic derivation in Appendix
\ref{complete-algebreic-derivation-of-value-creation-equation}.

\begin{align}
\nonumber \text{Sharpe Ratio}^I &> \text{Sharpe Ratio}^0\\
\nonumber \dfrac{C\times (r_c-r_f)-P^{I} \times(1 + r_c)}{SD(X_{ret})} &> \dfrac{(C-C_{x}^{0})\times (r_c-r_f)}{SD(X)}\\
\cdots &> \cdots\\
P^{I} &< \dfrac{(r_c-r_f)}{(1 + r_c)} \times C_{x}^{0} + \dfrac{(r_c-r_f)}{(1 + r_c)} \times C_{w}^{0} \times \left[1 - \dfrac{SD(X_{ret})}{SD(X)}\right] \label{final}
\end{align}

The interpretation of equation (\ref{final}) is slightly more difficult
than that posed by equation (\ref{eqn:cap-deploy}). However, we can
recognize that the first term on the righthand side is a measure of the
value of capital deployment from equation (\ref{eqn:cap-deploy}). Then
the second term is the value provided by the reduction in risk
volatility, and we intuitively understand that
\((1 - \dfrac{SD(X_{ret})}{SD(X)})\) is a representation of that
reduction. Further, it is intuitive that the reduced reduction in risk
would apply to the excess returns on working capital, i.e.,
\(((1 + r_c) \times C_{w}^{0})\).

We refer to equation (\ref{final}) as the ``Complete Value Creation
Equation'' since it includes \emph{both} sources of value. We should not
find this surprising since the Sharpe Ratio includes expected returns in
its numerator and risk/volatility in the denominator.

\hypertarget{conclusion-and-summary}{%
\section{Conclusion and Summary}\label{conclusion-and-summary}}

Through the analysis presented in the paper, we have developed a
measurement of two sources of value from the insurance transaction.

\begin{itemize}
\tightlist
\item
  The value that results from the ability of the firm to deploy
  additional capital:
\end{itemize}

\(\dfrac{(r_c-r_f)}{(1 + r_c)} \times C_{x}^{0}\)

When premiums amounts satisfy equation (\ref{eqn:cap-deploy}), the
insurance transaction will be accretive to the rate of return.

\begin{itemize}
\tightlist
\item
  The value created through the reduced earnings expectations that
  result from the reduction in volatility:
\end{itemize}

\(\dfrac{(r_c-r_f)}{(1 + r_c)} \times C_{w}^{0} \times \left[1 - \dfrac{SD(X_{ret})}{SD(X)}\right]\)

When premiums amounts do not satisfy equation (\ref{eqn:cap-deploy}) but
satisfy equation (\ref{final}), the insurance transaction will will not
be accretive to the rate of return. However, the reduced rate of return
will be less than the reduction in the rate of return demanded by the
shareholder.

When premiums amounts do not satisfy either equation
(\ref{eqn:cap-deploy}) or equation (\ref{eqn:cap-deploy}), then the
insurance transaction does not add financial value in excess of
premiums. However, it may still provide value in supporting a restful
night's sleep.

\newpage
\appendix

\hypertarget{complete-algebreic-derivation-of-value-creation-equations}{%
\section{Complete Algebreic Derivation of Value Creation
Equations}\label{complete-algebreic-derivation-of-value-creation-equations}}

\hypertarget{capital-deployment-value-creation-equation-1}{%
\subsection{Capital Deployment Value Creation
Equation}\label{capital-deployment-value-creation-equation-1}}

\begin{align}
\nonumber E[R^I] &> E[R^0] \\
\nonumber r_c \times (C - P^{I}) - P^{I} &> r_c \times C_{w}^{I} + r_f \times C_{x}^{0} \\
\nonumber r_c \times C - r_c \times P^{I} - P^{I} &>  r_c \times C_{w}^{I} + r_f \times C_{x}^{0}\\
\nonumber -P^{I} \times (1 + r_c) &> r_c \times C_{w}^{I} + r_f \times C_{x}^{0} - r_c \times C\\
\nonumber -P^{I} \times (1 + r_c) &> r_c \times (C_{w}^{I} - C) + r_f \times C_{x}^{0}\\
\nonumber -P^{I}  &> \dfrac{r_c \times (C_{w}^{I} - C) + r_f \times C_{x}^{0}}{1 + r_c}\\
\nonumber -P^{I} &> \dfrac{r_c \times (C_{w}^{I} - C) + r_f \times C_{x}^{0}}{1 + r_c}\\
\nonumber -P^{I} &> \dfrac{r_c \times (-C_{x}^{0}) + r_f \times C_{x}^{0}}{1 + r_c}\\
\nonumber -P^{I} &> C_{x}^{0} \times \dfrac{r_f - r_c}{1 + r_c}\\
P^{I} &< C_{x}^{0} \times \dfrac{r_c - r_f}{1 + r_c} \label{eqn:cap-deploy}
\end{align}

\hypertarget{complete-value-creation-equation-1}{%
\subsection{Complete Value Creation
Equation}\label{complete-value-creation-equation-1}}

\begin{align}
\nonumber \text{Sharpe Ratio}^I &> \text{Sharpe Ratio}^0\\
\nonumber \dfrac{C\times (r_c-r_f)-P^{I} \times(1 + r_c)}{SD(X_{ret})} &> \dfrac{(C-C_{x}^{0})\times (r_c-r_f)}{SD(X)}\\
\nonumber C\times (r_c-r_f)-P^{I} \times(1 + r_c) &> \dfrac{SD(X_{ret})}{SD(X)} \times (C-C_{x}^{0})\times (r_c-r_f)\\
\nonumber -P^{I} \times(1 + r_c) &> \dfrac{SD(X_{ret})}{SD(X)}\times (C-C_{x}^{0}) \times (r_c-r_f)\\
\nonumber &\mathrel{\phantom{<}} \quad - C \times (r_c-r_f)\\
\nonumber -P^{I} \times(1 + r_c) &> (r_c-r_f) \times \left[\dfrac{SD(X_{ret})}{SD(X)}\times (C-C_{x}^{0}) - C\right]\\
\nonumber P^{I} &< \dfrac{(r_c-r_f)}{(1 + r_c)} \times \left[\dfrac{SD(X_{ret})}{SD(X)}\times (C_{x}^{0}- C) + C\right]\\
\nonumber P^{I} &< \dfrac{(r_c-r_f)}{(1 + r_c)} \times \left[\dfrac{SD(X_{ret})}{SD(X)}\times C_{x}^{0}\right.\\
\nonumber &\mathrel{\phantom{<}}\left. \quad - \dfrac{SD(X_{ret})}{SD(X)}\times C + C\right]\\
\nonumber P^{I} &< \dfrac{(r_c-r_f)}{(1 + r_c)} \times \left[\dfrac{SD(X_{ret})}{SD(X)}\times C_{x}^{0}\right.\\
\nonumber &\mathrel{\phantom{<}} \left. \quad - \dfrac{SD(X_{ret})}{SD(X)}\times C + C_{x}^{0} + C_{w}^{0}\right]\\
\nonumber P^{I} &< \dfrac{(r_c-r_f)}{(1 + r_c)} \times C_{x}^{0} + \dfrac{(r_c-r_f)}{(1 + r_c)} \times \left[\dfrac{SD(X_{ret})}{SD(X)}\times C_{x}^{0}\right.\\
\nonumber &\mathrel{\phantom{<}}\left. \quad - \dfrac{SD(X_{ret})}{SD(X)}\times C + C_{w}^{0}\right]\\
\nonumber P^{I} &< \dfrac{(r_c-r_f)}{(1 + r_c)} \times C_{x}^{0} + \dfrac{(r_c-r_f)}{(1 + r_c)} \times \left[\dfrac{SD(X_{ret})}{SD(X)}\times (C - C_{w}^{0})\right.\\
\nonumber &\mathrel{\phantom{<}}\left. \quad - \dfrac{SD(X_{ret})}{SD(X)}\times C + C_{w}^{0}\right]\\
\nonumber P^{I} &< \dfrac{(r_c-r_f)}{(1 + r_c)} \times C_{x}^{0} + \dfrac{(r_c-r_f)}{(1 + r_c)} \times \left[\dfrac{SD(X_{ret})}{SD(X)}\times (-C_{w}^{0})  + C_{w}^{0}\right]\\
P^{I} &< \dfrac{(r_c-r_f)}{(1 + r_c)} \times C_{x}^{0} + \dfrac{(r_c-r_f)}{(1 + r_c)} \times C_{w}^{0} \times \left[1 - \dfrac{SD(X_{ret})}{SD(X)}\right] \label{final}
\end{align}

\newpage

\hypertarget{literature-review}{%
\section{Literature Review}\label{literature-review}}

We identified and reviewed the following papers in our literature
review. As noted, these papers focused on the return on the insuer's
capital rather than the value of capital preservation for the insured.

\href{https://www.insurancejournal.com/magazines/mag-closingquote/2015/01/12/353262.htm}{Macalaster,
Spencer. Insurance as a Form of Capital. January 12, 2015.} Described
the issue conceptually but without measurement and does not mention the
role of the actuary in establishing reserves.

\href{https://content.ebscohost.com/ContentServer.asp?T=P\&P=AN\&K=5054063\&S=R\&D=bth\&EbscoContent=dGJyMNXb4kSeqa440dvuOLCmsEieqK5Ss664TbOWxWXS\&ContentCustomer=dGJyMPGut1C0rLNRuePfgeyx83vt5OyF39\%2Fs}{Arrow,
K.J. Essays in the Theory of Risk Bearing; Markham: Chicago, IL, USA,
1971.} Arrow's theorem is the cornerstone of the application of
expected-ulitily theory in insurance

\href{https://www.jstor.org/stable/1802499}{Raviv, A. The design of
optimal insurance policy. Am. Econ. Rev.~1979, 69, 84--96.} Used
expected utility theory to measure value for an insured.

\href{https://www.jstor.org/stable/253617}{Gollier, C. Optimum insurance
of approximate losses. J. Risk Insur. 1996, 63, 369--380.} Also used
expected utility theory to measure optimize the insured retentions.

\href{https://www.jstor.org/stable/1911158}{Yaari, M. The dual theory of
choice under risk. Econometrica 1987, 55, 95--115.} The modification of
expected utility theory

\href{https://www.jstor.org/stable/41953283}{Kai A. Konrad; Stegrios
Skaperds. Self-Insurance and Self-Protection: A Nonexpected Utility
Analysis. The Geneva Papers on Risk and Insurance Theory, 1993-12,
Vol.18 (2), p.131-146.} Used rank-dependent expected utility preferences
to measure value for an insured

\href{https://www.researchgate.net/publication/5221351_Self-Insurance_Self-Protection_and_Market_Insurance_within_the_Dual_Theory_of_Choice_In_The_Geneva_Papers_on_Risk_and_Insurance_Theory_26_1_43-56}{Christophe
Courbage. Self-Insurance, Self-Protection and Market Insurance within
the Dual Theory of Choice. The Geneva papers on risk and insurance
theory, 2001-06-01, Vol.26 (1), p.43-56.} Used dual theory to measure
value for an insured

\href{https://www.mdpi.com/2227-9091/2/2/226/htm}{Michael Merz; Mario V.
Wüthrich, Demand of Insurance under the Cost-of-Capital Premium
Calculation Principle. Risks, 2014, Vol.2(2), 226-248.} Used risk
premium approach to measure value for an insured

\href{https://www.researchgate.net/publication/228293026_Optimal_Insurance_Design_Under_Rank-Dependent_Expected_Utility}{Bernard,
C.; He, X.D.; Yan, J.-A.; Zhou, X.Y. Optimal insurance design under
rank-dependent expected utility. In Mathematical Finance; SSRN Preprint:
Waterloo, Canada, 2014.} Used rank-dependent expected utility theory to
measure value for an insured too

\href{https://www.researchgate.net/publication/225581818_Insurance_Decisions_for_Low_Probability_Losses}{Laury,
S.; M. Mcinnes, and J.Swarthout(2009). Insurance Decisions for
Low-Probability Losses. Journal of Risk and Uncertainty, 39, 17--44.}
Used experimental evidence to show individual's under-insure decision
for low-probability high-loss events

\href{https://www.jstor.org/stable/41761127}{Kunreuther, H.; Pauly, M.
(2004). Neglecting disaster: Why don't people insure against large
losses? Journal of Risk and Uncertainty, 28, 5--21.} Used utility theory
to show the reason why individuals don't insure low-probability and
high-loss events

\end{document}
